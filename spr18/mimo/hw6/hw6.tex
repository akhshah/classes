\documentclass{article}
\usepackage[margin=1in]{geometry}
%\usepackage[numbered, framed,autolinebreaks,useliterate]{mcode}
\usepackage{amsmath, amsthm}
\usepackage{amssymb}
\usepackage{graphicx}
\usepackage{epstopdf}
\usepackage{cancel}
\usepackage{subcaption}
\usepackage{mathtools}
\usepackage{siunitx}
\sisetup{output-exponent-marker=\ensuremath{\mathrm{e}}}
\usepackage{float}
\usepackage[T1]{fontenc}
\usepackage[numbered, framed]{matlab-prettifier}
\usepackage{pythontex}

\lstset{
  style              = Matlab-editor,
  basicstyle         = \mlttfamily,
  escapechar         = ",
  mlshowsectionrules = true,
}

\begin{document}
\begin{center}
	{\huge ASE 381P: Assignment 6}
\end{center}
\begin{center}
	Akhil Shah
\end{center}
\begin{center}
	March 20th, 2018
\end{center}

\noindent
{\large Problem 1}
\newline 
In this problem, the largest decay rate is when $\alpha = 1$ which corresponds to the maximum negative eigenvalue of A. 
\lstinputlisting{./prob1.m}

\newpage
\noindent
{\large Problem 2}
\newline

A change of variables is required to allow for a $K$ to be found.
$$ A_{cl}^TP + PA_{cl} + 2\alpha P < 0 $$
$$ (A + BK)^TP + P(A + BK) + 2\alpha P < 0 $$
$$ A^TP + K^TB^TP + PA + PBK + 2\alpha P < 0 $$

However, it is impossible to find a K from the above equation and so pre- and post-multiply by $P^{-1}$. 
$$ P^{-1}A^T + P^{-1}K^TB^T + AP^{-1} + BKP^{-1} + 2\alpha P^{-1} < 0 $$

And set $Q = P^{-1}$ and $Z = KP^{-1}$. Which becomes:
$$ QA^T + Z^TB^T + AQ + BZ + 2\alpha Q < 0 $$
From the bisection method we get $\alpha = 25.512$ and $ K = 1e6 \cdot \begin{bmatrix}  6.5754 & 0.4491 & 6.5421 & 0.4412 \end{bmatrix} $. 

\lstinputlisting{./prob2.m}

\end{document}