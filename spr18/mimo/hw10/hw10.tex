\documentclass{article}
\usepackage[margin=1in]{geometry}
\usepackage{amsmath, amsthm}
\usepackage{amssymb}
\usepackage{graphicx}
\usepackage{siunitx}

% --- Begin Document
\begin{document}

% --- Title
\begin{center}
    {\huge Multivariate Control Systems - Homework 10}
\end{center}
\begin{center}
    Akhil Shah
\end{center}
\begin{center}
    April 20, 2018
\end{center}

\noindent
{\large Problem 1}
\newline
\newline
Show that 
\begin{equation}
    [I - G^{\sim}(s)G(s)] = \left[ \begin{array}{cc | c}
                           A & BB^{*} & B \\
                           -C^{*}C & -A^{*} & 0 \\
                           \hline
                           0 & B^{*} & I\\
                           \end{array} \right]
\end{equation}
Using the multiplication rules found in Chapter 2:
\begin{equation}
    G^{\sim}(s)G(s) = \left[ \begin{array}{cc | c}
                     -A^{*} & C^{*}C & 0 \\
                     0 & A & B \\
                     \hline
                     B^{*} & 0  & 0\\
                     \end{array} \right]
\end{equation}
And then the inversion rule:
\begin{align}
    A &= \begin{bmatrix} -A^{*} & C^{*}C \\
                        0 & A \\
        \end{bmatrix} - \begin{bmatrix} B \\ 0 \end{bmatrix} \begin{bmatrix} B^{*} & 0 \end{bmatrix} \\
    BD^{-1} &= \begin{bmatrix} B \\ 0 \end{bmatrix} \\
    -D^{-1}C &= \begin{bmatrix} B^* & 0 \end{bmatrix}
\end{align}
From this result and changing the order of the inputs we get:
\begin{equation}
    [I - G^{\sim}(s)G(s)] = \left[ \begin{array}{cc | c}
                           A & BB^{*} & B \\
                           -C^{*}C & -A^{*} & 0 \\
                           \hline
                           0 & B^{*} & I\\
                           \end{array} \right]
\end{equation}

\noindent
{\large Problem 2}
\newline
\newline
i) In this case $\begin{bmatrix} C_2 D_{21} \end{bmatrix} = \begin{bmatrix} I & 0 \end{bmatrix}$. The kernel is then:
\begin{equation} \begin{bmatrix} 0 \\ I \end{bmatrix} \end{equation}
Then utilizing Theorem 7.10:
\begin{equation}
    N_c\begin{bmatrix} A^{*}X + XA & XB_1 & C_1^* \\
                       B_1^*X & -I & D_{11}^* \\
                       C_1 & D_{11} & -I
       \end{bmatrix}N_c
\end{equation}
where $N_c$ is:
\begin{equation}
    N_c = \begin{bmatrix} 0 & 0 \\ I & 0 \\ 0 & I \end{bmatrix}
\end{equation}
And through judicious algebra we get the final result as:
\begin{equation}
    \begin{bmatrix} -I & D_{11}^* \\ D_{11} & -I \end{bmatrix} < 0
\end{equation}
ii) Using the elimination lemma, with $N_Q$ as:
\begin{equation}
    N_Q = \begin{bmatrix} 0 & 0 \\ I & 0 \\ 0 & I \end{bmatrix}
\end{equation}
and H as:
\begin{equation}
    H = \begin{bmatrix} AY + YA^* & YC_1^* & B_1 \\ C_1Y & -I & D_{11} \\ B_1^* & D_{11}^* & -I \end{bmatrix}
\end{equation}
which results in:
\begin{equation}
    \begin{bmatrix} -I & D_{11}^* \\ D_{11} & -I \end{bmatrix} < 0
\end{equation}
iii) With $u = Fx$
\begin{align*}
    \dot{x} &= (A + B_2F)x + B_1w \\
    z &= (C_1 + D_{12}F)x + D_{11} \\
    y &= x
\end{align*}
\begin{equation}
    \begin{bmatrix} C^* \\ D^* \end{bmatrix}\begin{bmatrix} C & D \end{bmatrix} + \begin{bmatrix} A^*X + XA & XB \\ B^*X & -I \end{bmatrix} < 0
\end{equation}
And then using the Schur complemenet we get:
\begin{equation}
    \begin{bmatrix}
        (A + B_2F)^*X + X(A + B_2F) & (C_1 + D_{12})^* & XB_1 \\
        C_1 + D_{12} & -I & D_{11} \\
        B_1^*X & D_{11}^* & -I
    \end{bmatrix}
\end{equation}
With a transformation of:
\begin{equation}
    \begin{bmatrix}
        X^{-1} & & \\
        & I & \\
        & & I
    \end{bmatrix}
\end{equation}
Which gives the final result.
\end{document}
